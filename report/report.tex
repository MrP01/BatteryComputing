\documentclass{prettytex/ox/mmsc-special-topic}
\usepackage{cancel}
\usepackage{empheq}
\usepackage{tikz-3dplot}
\usepackage{tabularray}
\usetikzlibrary{3d}

\setlength{\headheight}{19.53pt}
\setlength{\headsep}{1.2em}
\setlength{\belowcaptionskip}{-12pt}
\setminted{fontsize=\footnotesize}
\AfterEndEnvironment{minted}{\vspace*{-0.8cm}}
\renewcommand{\operatorcolor}{black}

\newcommand{\chebyshev}{Chebyshev\xspace}
\newcommand{\tschebfun}{\textcolor{themecolor3}{TschebFun}\xspace}
\newcommand{\heatfun}{\textcolor{themecolor3}{HeatFun}\xspace}

\tikzset{
  paper/.style={draw, fill=none},
  irregular border/.style={decoration={irregular fractal line, amplitude=0.2},
      decorate,
    },
  ragged border/.style={ decoration={random steps, segment length=7mm, amplitude=2mm},
      decorate,
    }
}

\newcommand{\laplace}{\mbox{\setlength{\unitlength}{0.1em}%
    \begin{picture}(20,10)%
      \put(2,3){\circle{4}}%
      \put(4,3){\line(1,0){13}}%
      \put(18,3){\circle*{4}}%
    \end{picture}%
  }%
}

\addbibresource{sources.bib}
\tikzexternalize[prefix=tikz/]

\newcommand{\topictitle}{Battery Computing}
\newcommand{\candidatenumber}{1072462}
\newcommand{\course}{Scientific Computing}

\title{\topictitle}
\author{Candidate \candidatenumber}
\date{\today}

\begin{document}
  \pagestyle{plain}
  \mmscSpecialHeader[casestudy]

  \begin{abstract}
    \label{abstract}
    This work shall attempt to
    \vspace*{0.2cm}

    \noindent
    \textbf{Our Goal:}
    Numerically obtain the solution $\{a(x, T), b(x, T)\}$ of
    \vspace*{-0.2cm}
    $$\begin{cases}
        % \setstretch{1}
        \frac{\partial a}{\partial t} = D_a \frac{\partial^2 a}{\partial x^2}, & a: \R^+ \times [0, T] \mapsto [0, 1],\, T \in \R^+,\; D_a \in \R^+, \\[-0.2em]
        \frac{\partial b}{\partial t} = D_b \frac{\partial^2 b}{\partial x^2}, & b: \R^+ \times [0, T] \mapsto [0, 1],\, D_b \in \R^+,               \\[-0.2em]
        a(\infty, t) = 1,\; b(\infty, t) = 0                                   & \forall t \in [0, T]                                                \\[-0.2em]
        a(x, 0) = 1,\;\; b(x, 0) = 0                                           & \forall x \in (0, \infty)                                           \\[-0.2em]
        a(0, t) = 0,\; \frac{\partial a}{\partial x} + D \frac{\partial b}{\partial x} = 0
      \end{cases}$$
    \vspace*{0.05cm}

    The implementation bla bla
  \end{abstract}

  \begin{figure}[H]
    \centering
    \includegraphics[width=\linewidth]{figures/screenshot.png}
    \caption{Graphical User Interface}
  \end{figure}

  \pagebreak
  \pagestyle{normal}

  \tableofcontents
  \pagebreak

  \section{Problem Introduction}
  Clearly, batteries relevant.

  \begin{figure}[H]
    \centering
    Figure
    \caption{Wohoo}
    \label{fig:fig}
  \end{figure}

  \subsection{Chronoamperometry}
  \subsection{DC Voltammetry}
  \subsection{AC Voltammetry}

  \section{Mathematical Background}
  Let $\N$ denote the nonnegative integers, so $0 \in \N$.
  Similarly, let $\R^+ = [0, \infty)$ denote the nonnegative real numbers.
  \autoref{fig:fig}.

  \subsection{Laplace Integral Transform}
  What is Laplace?

  Proof for Laplace's differentiation theorem.

  \subsection{Chebyshev Polynomials}
  Proof of $U_k(-1)$'s value.

  \section{Finite Differences}
  Construct $A \vec{x} = \vec{b}$.

  \subsection{Results}

  \section{Analytical Approaches}
  \subsection{Similarity Solution}
  \subsection{Integral Equation}
  \subsubsection{Derivation}
  \subsubsection{Numerical Solution}

  \section{Spectral Method}
  From the definition of Chebyshev polynomials $T_k(x) = \cos(k\theta)$, we can derive that
  $$\frac{\dd T_k}{\ddx} = \frac{\dd T_k}{\dd\theta} \frac{\dd\theta}{\ddx} = ... = k U_{k-1}(x)\,,$$
  where $U_k: [-1, 1] \mapsto \R$ denote the Chebyshev polynomials of the second kind, which in turn are defined by
  $$U_k(\cos \theta) \sin(\theta) = \sin\left((n+1) \theta\right)\,.$$

  In order to enforce a von-Neumann boundary condition on the left and a Dirichlet boundary condition on the right,
  we are interested in explicitly setting coefficients $a_k$ such that
  $$a_x(-1, t) = \frac{\dd a}{\ddx}\Big|_{x=-1} = \tilde{l} \quad \text{ and } \quad a(1) = r, \quad \text{ where } \quad \tilde{l}, r \in \R\,.$$

  Using the Chebyshev series ansatz
  $$a(x, t) = \sum_{k=0}^{N-1} a_k^{(t)} T_k(x)$$
  we have that
  $$\frac{\dd a}{\ddx} = \sum_{k=0}^{N-1} a_k^{(t)} \frac{\dd T_k}{\ddx}(x)\,,$$
  so we are interested in
  $$a_x(-1, t) = \frac{\dd a}{\ddx}\Big|_{x=-1} = \sum_{k=0}^{N-1} a_k^{(t)} \frac{\dd T_k}{\ddx}\Big|_{x=-1} = \sum_{k=0}^{N-1} a_k^{(t)} k U_{k-1}(-1)\,.$$

  Following from TODO (explained on Wikipedia), we know that
  $$U_k(-1) = (-1)^k (k+1) \quad \text{ and } \quad T_k(1) = 1 \quad \forall k \in \N\,,$$
  which turns our conditions into algebraic conditions w.r.t. the coefficients $a_k^{(t)}$,
  $$a_x(-1, t) = \frac{\dd a}{\ddx}\Big|_{x=-1} = \sum_{k=0}^{N-1} a_k^{(t)} k^2 (-1)^{k-1} \overset{!}{=} \tilde{l} \quad \text{ and } \quad a|_{x=1} = \sum_{k=0}^{N-1} a_k^{(t)} \overset{!}{=} r\,.$$

  Knowing that the heat equation Forward Euler numerical scheme modifies all but the two highest-degree coefficients in the series, we expand:
  \begin{align*}
    a_x(-1, t) & = \sum_{k=0}^{N-1} a_k^{(t)} T_k'(-1) & = & \overbrace{-\sum_{k=0}^{N-3} a_k^{(t)} k^2 (-1)^{k}}^{:= \Sigma_3} & - & (N-2)^2 (-1)^{N-2} a_{N-2} & - & (N-1)^2 (-1)^{N-1} a_{N-1} & = l\,, \\
    a(1, t)    & = \sum_{k=0}^{N-1} a_k^{(t)} T_k(1)   & = & \underbrace{\sum_{k=0}^{N-3} a_k^{(t)}}_{:= \Sigma_2}              & + & a_{N-2}                    & + & a_{N-1}                    & = r\,,
  \end{align*}

  \subsection{Enforcing Boundary Conditions}
  Von Neumann on the left

  \subsection{Implicit Euler}

  \subsection{Implementation}
  The solver was implemented in C++.

  % \inputminted{cpp}{../SpectralSolver/Solver.h}

  \subsection{Results}
  Chronoamperometry,
  DC Voltammetry,
  AC Voltammetry

  \section{Conclusion}
\end{document}
